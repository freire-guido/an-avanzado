\documentclass{article}
\usepackage{ntheorem}
\usepackage{enumitem}
\usepackage{amsmath}
\usepackage{amsfonts}
\usepackage[spanish]{babel}

\title{An\'alisis Avanzado}
\author{Guido Freire}
\date{2C 2022}

\theoremstyle{break}
\theorembodyfont{\upshape}
\newtheorem{ejercicio}{Ejercicio}[section]
\newtheorem*{solucion}{Soluci\'on}
\newtheorem*{teorema}{Teorema}
\newtheorem*{demostracion}{Demostraci\'on}

\newcommand{\enum}[1]{\begin{enumerate}[label=(\alph*)] #1 \end{enumerate}}
\newcommand{\enumr}[1]{\begin{enumerate}[label=(\roman*)] #1 \end{enumerate}}



\title{An\'alisis Avanzado}
\author{Guido Freire}
\date{2C 2022}

\begin{document}
\maketitle

\section{Secuencias}
\begin{definicion}[Supremo]
    $s = sup(A) \iff \forall \epsilon > 0,\, \exists a \in A /\; s - \epsilon < a$
\end{definicion}
\begin{definicion}[Convergencia]
    \begin{align*}
        x_n \rightarrow x &\iff \forall \epsilon > 0,\, \exists N \in \mathbb{N} /\; n \geq N \implies d(x_n, x) < \epsilon\\
            &\iff \forall B_\epsilon(x),\, \exists N \in \mathbb{N} /\; n \geq N \implies x_n \in B_\epsilon(x)\\
            &\implies \{x_n\} \text{ es acotada}\\
            & \impliedby \{x_n\} \text{ es acotada y mon\'otona}
    \end{align*}
\end{definicion}
\begin{teorema}[\'Algebra de L\'imites]
    \begin{itemize}
    \item $a_n + b_n \rightarrow a + b$
    \item $a_n\cdot b_n \rightarrow a\cdot b$
    \item $a_n / b_n \rightarrow a/b$
    \end{itemize}
\end{teorema}
\begin{teorema}[Ordenes de L\'imites]
    \begin{itemize}
        \item $a_n \leq b_n \implies a \leq b$
        \item $a_n \leq c \implies a \leq c$
        \end{itemize}
\end{teorema}
\begin{teorema}[Subsecuencias]
    \begin{align*}
        x_n \rightarrow x &\implies \forall x_{n_k} \rightarrow x\\
        \{x_n\} \text{ acotada} &\implies \exists x_{n_k} \rightarrow x
    \end{align*}
\end{teorema}

\section{Cardinalidad}
\begin{definicion}[Contable]
    $A$ contable $\iff A$ finito o numerable.
\end{definicion}
\begin{teorema}
    $A$ infinito $\implies \exists B \subseteq A$ infinito numerable.
\end{teorema}
\begin{teorema}
    $A$ contable $\implies \forall B \subseteq A$ contable o finito.
\end{teorema}
\begin{definicion}
    $\#A \leq \#B \iff \exists f : A \rightarrow B$ inyectiva.
\end{definicion}
\begin{teorema}[Schröder-Bernstein]
    $\#A \leq \#B \land \#B \leq \#A \implies \#A = \#B$\\
    $\exists f,g /\; f: A \rightarrow B,\, g: B \rightarrow A$ inyectivas $\iff \exists h : A \rightarrow B$ biyectiva.
\end{teorema}
\begin{teorema}[Cantor]
    $\#A < \#\mathcal{P}(A)$
\end{teorema}
\begin{teorema}
    $A_n$ numerable $\implies A = \bigcup_{n\in\mathbb{N}} A_n$ es numerable
\end{teorema}

\section{Espacios M\'etricos}

\section{Continuidad}

\section{Compacidad}

\end{document}