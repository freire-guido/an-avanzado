\documentclass{book}
\usepackage{ntheorem}
\usepackage{enumitem}
\usepackage{amsmath}
\usepackage{amsfonts}
\usepackage[spanish]{babel}

\title{An\'alisis Avanzado}
\author{Guido Freire}
\date{2C 2022}

\theoremstyle{break}
\theorembodyfont{\upshape}
\newtheorem{ejercicio}{Ejercicio}[section]
\newtheorem*{solucion}{Soluci\'on}
\newtheorem*{teorema}{Teorema}
\newtheorem*{demostracion}{Demostraci\'on}

\newcommand{\enum}[1]{\begin{enumerate}[label=(\alph*)] #1 \end{enumerate}}
\newcommand{\enumr}[1]{\begin{enumerate}[label=(\roman*)] #1 \end{enumerate}}



\title{An\'alisis Avanzado}
\author{Guido Freire}
\date{2C 2022}

\begin{document}

\maketitle

\chapter{Primer Parcial}

\section{Practica 1}
\begin{teorema}[Heine-Borel]
        Un conjunto $K \subseteq \mathbb{R}$ es \textit{compacto} si:
        \enum{
            \item Es cerrado y acotado.
            \item Toda secuencia en $K$ tiene una subsecuencia que converge en $K$.
            \item Todo cubrimiento abierto de $K$ tiene un subcubrimiento finito.
        }    
\end{teorema}
\begin{definicion}[Perfecto]
    Un conjunto $P \subseteq \mathbb{R}$ es \textit{perfecto} si es cerrado y no contiene puntos aislados.
    (Todo $x \in P$ es un punto limite).
\end{definicion}
\beginc{ejercicio}{15}
    \enum{
        \item Si $\lim\limits_{k\to\infty} x_{2k} = \lim\limits_{k\to\infty}  x_{2k-1}$ entonces $(x_n)_{n\in\mathbb{N}}$ es convergente.
        \item Si $(x_{2k})_{k\in\mathbb{N}}$, $(x_{2k-1})_{k\in\mathbb{N}}$ y $(x_{3k})_{k\in\mathbb{N}}$ son convergentes entonces $(x_n)_{n\in\mathbb{N}}$ es convergente.
    }
\end{ejercicio}
\begin{solucion}
    \enum{
        \item Existen $k_{2k}$, $k_{2k-1}$ que garantizan $\forall \epsilon > 0$ que $|x_{2k} - c| < \epsilon$, $|x_{2k-1} - c| < \epsilon$
            \[ |x_{2k} + x_{2k-1}| \leq |x_{2k} + c| + |c - x_{2k + 1}| < 2\epsilon\]
            Basta tomar un $k_0 = \max(k_{2k}, k_{2k-1})$ para garantizar $|x_n - x_m| < \epsilon$.\par
            En general dado $A = \bigcup\limits_{i}^n A_i$, $(x_a)_{a\in A}$ converge a $c \iff$ $(x_a)_{a\in A_i}$ convergen a $c$.
        \item Las subsecuencias convergentes de una secuencia convergente comparten limite. Como $(x_{3k}) \cap (x_{2k})$ es una subsecuencia de $(x_{2k})$ y $(x_{3k})$ que converge a $c$, $(x_{2k})$ y $(x_{3k})$ tambi\'en lo hacen.\\
            Usando un razonamiento analogo con $(x_{3k}) \cap (x_{2k-1})$ se deduce que $(x_{2k})$ y $(x_{2k + 1})$ convergen a $c$. Se reduce a (a).
    }
\end{solucion}
\setcounter{section}{2}
\section{Practica 3}
\begin{definicion}[Punto interior]
    Dado $A \subseteq X$, $x \in A$ se dice \textit{punto interior} $\iff \exists B_r(x) \subset A$
\end{definicion}
\begin{definicion}[Punto aislado]
    Dado $A \subseteq X$, $x \in A$ se dice \textit{punto aislado} $\iff \exists B_r(x) \cap A = \{x\}$
\end{definicion}
\begin{definicion}[Punto de adherencia]
    Dado $A \subseteq X$, $x \in X$ se dice \textit{punto de adherencia} $\iff \forall B_r(x) \cap A \neq \emptyset$
\end{definicion}
\begin{definicion}[Punto de acumulaci\'on]
    Dado $A \subseteq X$, $x \in X$ se dice \textit{punto de acumulacion} $\iff \forall B_r(x)-\{x\} \cap A \neq \emptyset$
\end{definicion}
\section{Practica 4}
\beginc{ejercicio}{7}
    \enum{
        \item $\{(x, y) \in \mathbb{R}^2 : x^2 + y\sin(e^x-1)=-2\}$ es cerrado.
        \item $\{(x, y, z) \in \mathbb{R}^3 : -1 \leq x^3 -3y^4 + z - 2 \leq 3\}$ es cerrado.
        \item $\{(x_1, x_2, x_3, x_4, x_5) \in \mathbb{R}^5 : 3 < x_1 - x_2\}$ es abierto.
    }
\end{ejercicio}
\begin{solucion}
    \enum{
        \item $A$ es cerrado $\iff A^c$ es abierto.\\
        Construyo una $\{x_n, y_n\} \subset A^c / \{x_n, y_n\} \rightarrow A$.
        
    }
\end{solucion}

\chapter{Segundo Parcial}



\chapter{Final}

\begin{teorema}[$L^1$ es completo]
    Sea $(f_n)_{n\in\mathbb{N}} \in L^1$ de Cauchy. Tomar $\epsilon_k = \frac{1}{2^k} \Rightarrow \exists N_0 \cdots N_k \,/\, || f_{n_{k+1}} - f_{n_k} || < \epsilon_k$
    \begin{align*}
        f &= f_{n_1} + \sum_{k=1}^{\infty}(f_{n_{k+1}} - f_{n_k}) \leq |f|\\
        \int |f| &\leq \int |f_{n_1}| + \int \sum_{k=1}^{\infty}|f_{n_{k+1}} - f_{n_k}| \quad\text{(mon\'otona)}\\
                 &= \int |f_{n_1}| + \sum_{k=1}^{\infty} ||f_{n_{k+1}} - f_{n_k}|| \leq \int |f_{n_1}| + \sum_{k=1}^{\infty} \frac{1}{2^k}
    \end{align*}
    As\'i que $f \in L^1$. Queda confirmar $f_{n_k} \to f$. Una forma es notando:
        $$|f - f_{n_K}| \leq \sum_{k=K}^{\infty}|f_{n_{k+1}} - f_{n_k}| \to 0$$
    Que integrando y usando convergencia mon\'otona concluye que converge en $L^1$. Una m\'as en l\'inea con lo que viene es:
    \begin{align*}
        |f - f_{n_K}| &\leq ||f| + |f_{n_K}|| \leq 2|f| \quad \text{(mayorada)}\\
        \lim_{K\to\infty}\int|f - f_{n_K}|\, d\mu &= \int\lim_{K\to\infty}|f - f_{n_K}|\, d\mu = 0
    \end{align*}
\end{teorema}
\begin{teorema}[$L^2$ tiene un producto interno]
    \enum{
    \item Cauchy-Schwarz: Asumir $||f'||, ||g'|| \neq 0$, tomar $f' = f/||f||$ y $g' = g/||g||$
        \begin{align*}
            0 &\leq |f|^2 - 2|fg| + |g|^2 = (|f|-|g|)^2\\
            2fg &\leq |f|^2 + |g|^2\\
            2\int fg &\leq ||f||^2 + ||g||^2
        \end{align*}
    \item Desigualdad Triangular
        $$||f + g||^2 = (f + g,\, f + g) \leq ||f||^2 + 2||f||\,||g|| + ||g||^2 \leq (||f|| + ||g||)^2$$
    \item Pit\'agoras
        $$||f + g||^2 = ||f||^2 + ||g||^2 \quad (f \perp g)$$
    }
\end{teorema}
\begin{teorema}[$B(X)$ es denso en $L^2(X)$]
    Dada $f \in L^2(X)$, tomar la funci\'on acotada
    $$f_n(x) =
    \begin{cases}
        f(x)    & |f(x)| \leq n\\
        0       & |f(x)| > n
    \end{cases}
    $$
    Como $|f(x) - f_n(x)|^2 \leq 4|f(x)|^2$, est\'a mayorada, as\'i que:
    $$
    \lim_{n\to\infty}||f - f_n|| = \lim_{n\to\infty} \int_X |f(x) - f_n(x)|^2\,dx = 0
    $$
\end{teorema}

\chapter{Otros}
\section{Digresionario}
\begin{teorema}[Bolas de Brouwer]
    Dado un conjunto $K$ y una funcion $f$ satisfaciendo las condiciones del teorema del punto fijo de Brouwer,
    es corolario el teorema de las \textit{bolas peludas}:
    \enum{
        \item Construyo un espacio vectorial \textit{(posible per Brouwer)}.
        \item Como hay un punto $x \in K: x = f(x)$, 
    }
\end{teorema}

\end{document}
