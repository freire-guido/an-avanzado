\documentclass{book}
\usepackage{ntheorem}
\usepackage{enumitem}
\usepackage{amsmath}
\usepackage{amsfonts}
\usepackage[spanish]{babel}

\title{An\'alisis Avanzado}
\author{Guido Freire}
\date{2C 2022}

\theoremstyle{break}
\theorembodyfont{\upshape}
\newtheorem{ejercicio}{Ejercicio}[section]
\newtheorem*{solucion}{Soluci\'on}
\newtheorem{teorema}{Teorema}[section]
\newtheorem{definicion}{Definicion}[section]
\newtheorem*{demostracion}{Demostraci\'on}

\newcommand{\enum}[1]{\begin{enumerate}[label=(\alph*)] #1 \end{enumerate}}
\newcommand{\enumr}[1]{\begin{enumerate}[label=(\roman*)] #1 \end{enumerate}}

\newcommand{\beginc}[2]{\setcounter{#1}{\numexpr #2 - 1} \begin{#1}}

\begin{document}

\maketitle

\chapter{Primer Parcial}
\section{Practica 1}
\begin{teorema}[Heine-Borel]
        Un conjunto $K \subseteq \mathbb{R}$ es \textit{compacto} si:
        \enum{
            \item Es cerrado y acotado.
            \item Toda secuencia en $K$ tiene una subsecuencia que converge en $K$.
            \item Todo cubrimiento abierto de $K$ tiene un subcubrimiento finito.
        }    
\end{teorema}
\begin{definicion}[Perfecto]
    Un conjunto $P \subseteq \mathbb{R}$ es \textit{perfecto} si es cerrado y no contiene puntos aislados.
    (Todo $x \in P$ es un punto limite).
\end{definicion}
\beginc{ejercicio}{16}
    \enum{
        \item Si $\lim\limits_{k\to\infty} x_{2k} = \lim\limits_{k\to\infty}  x_{2k+1}$ entonces $(x_n)_{n\in\mathbb{N}}$ es convergente.
        \item Si $(x_{2k})_{k\in\mathbb{N}}$, $(x_{2k+1})_{k\in\mathbb{N}}$ y $(x_{3k})_{k\in\mathbb{N}}$ son convergentes entonces $(x_n)_{n\in\mathbb{N}}$ es convergente.
    }
\end{ejercicio}
\begin{solucion}
    \enum{
        \item Existen $k_{2k}$, $k_{2k+1}$ que garantizan $\forall \epsilon > 0$ que $|x_{2k} - c| < \epsilon$, $|x_{2k+1} - c| < \epsilon$
            \[ |x_{2k} + x_{2k+1}| \leq |x_{2k} + c| + |c - x_{2k + 1}| < 2\epsilon\]
            Basta tomar un $k_0 = \max(k_{2k}, k_{2k+1})$ para garantizar $|x_n - x_m| < \epsilon$.\par
            En general dado $A = \bigcup\limits_{i}^n A_i$, $(x_a)_{a\in A}$ converge a $c \iff$ $(x_a)_{a\in A_i}$ convergen a $c$.
        \item Las subsecuencias convergentes de una secuencia convergente comparten limite. Como $(x_{3k}) \cap (x_{2k})$ es una subsecuencia de $(x_{2k})$ y $(x_{3k})$ que converge a $c$, $(x_{2k})$ y $(x_{3k})$ tambi\'en lo hacen.\\
            Usando un razonamiento analogo con $(x_{3k}) \cap (x_{2k+1})$ se deduce que $(x_{2k})$ y $(x_{2k + 1})$ convergen a $c$. Se reduce a (a).
    }
\end{solucion}
\setcounter{section}{3}
\section{Practica 4}
\beginc{ejercicio}{7}
    \enum{
        \item $\{(x, y) \in \mathbb{R}^2 : x^2 + y\sin(e^x-1)=-2\}$ es cerrado.
        \item $\{(x, y, z) \in \mathbb{R}^3 : -1 \leq x^3 -3y^4 + z - 2 \leq 3\}$ es cerrado.
        \item $\{(x_1, x_2, x_3, x_4, x_5) \in \mathbb{R}^5 : 3 < x_1 - x_2\}$ es abierto.
    }
\end{ejercicio}
\begin{solucion}
    \enum{
        \item $A$ es cerrado $\iff A^c$ es abierto.\\
        Construyo una $\{x_n, y_n\} \subset A^c / \{x_n, y_n\} \rightarrow A$.
        
    }
\end{solucion}

\chapter{Segundo Parcial}
\chapter{Otros}
\section{Explicario}
\begin{teorema}[Bolas de Brouwer]
    Dado un conjunto $K$ y una funcion $f$ satisfaciendo las condiciones del teorema del punto fijo de Brouwer,
    es corolario el teorema de las \textit{bolas peludas}:
    \enum{
        \item Construyo un espacio vectorial \textit{(posible per Brouwer)}.
        \item Como hay un punto $x \in K: x = f(x)$, 
    }
\end{teorema}

\end{document}
